\documentclass[conference]{IEEEtran}

\title{Native, Non-Native, and Cross-Platform Mobile Application Development: A Comparative Analysis}
\author{\IEEEauthorblockN{Castillo Mendez Josue Leonel}
	\IEEEauthorblockA{UTT\\
		0322103691@ut-tijuana.edu.mx}}
\date{}

\begin{document}
	
	\maketitle
	
	\begin{abstract}
		This paper conducts a thorough examination of mobile application development approaches, comparing and contrasting native, non-native, and cross-platform methodologies. The study explores the differences, advantages, and disadvantages inherent in each approach.
	\end{abstract}
	
	\section{Introduction}
	The landscape of mobile application development is diverse, with developers having to choose between native, non-native, and cross-platform approaches. This paper aims to provide a comprehensive analysis of the differences, advantages, and disadvantages associated with each methodology.
	
	\section{Native, Non-Native, and Cross-Platform Development}
	\subsection{Definitions and Distinctions}
	Define each approach, elucidating the fundamental differences in their development paradigms.
	
	\subsection{Advantages of Native Development}
	Explore the benefits of native application development, including optimized performance and access to platform-specific features.
	
	\subsection{Disadvantages of Native Development}
	Examine the drawbacks such as longer development times and increased costs associated with native development.
	
	\subsection{Advantages of Non-Native Development}
	Investigate the advantages of non-native development, emphasizing cost-effectiveness and faster development cycles.
	
	\subsection{Disadvantages of Non-Native Development}
	Discuss potential limitations, such as compromised performance and less access to platform-specific features.
	
	\subsection{Advantages and Challenges of Cross-Platform Development}
	Analyze the benefits of cross-platform development, including code reusability, and address challenges like potential performance issues and compatibility concerns.
	
	\section{Case Studies and Comparative Analysis}
	\subsection{Native Applications}
	Provide case studies illustrating successful native application development, highlighting the applications' performance and user experience.
	
	\subsection{Non-Native Applications}
	Present examples of non-native applications, showcasing their development efficiency and trade-offs in performance.
	
	\subsection{Cross-Platform Applications}
	Explore cross-platform application case studies, focusing on their ability to reach a broader audience and potential challenges.
	
	\section{Conclusion}
	In conclusion, the choice between native, non-native, and cross-platform development depends on various factors, including project requirements, budget constraints, and performance expectations. This comparative analysis aims to assist developers in making informed decisions based on the specific needs of their projects.
	
	\section{References}
	\begin{thebibliography}{1}
		\bibitem{emma} L. Nunez, “Tipos de aplicaciones, características, ejemplos y comparativa | EMMA,” | EMMA, Jan. 04, 2023. \textless https://emma.io/blog/tipos-aplicaciones-caracteristicas-ejemplos/\textgreater.
		
		\bibitem{pixzelle} P. S. S. de R. de CV, “Agencia de desarrollo de apps móviles y más!,” Pixzelle Studio. \textless https://www.pixzelle.mx/blog/apps-nativas-vs-multiplataforma\textgreater.
	\end{thebibliography}
	
\end{document}

